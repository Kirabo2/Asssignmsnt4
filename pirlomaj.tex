\documentclass[9pt,a4paper]{article}

\usepackage{zed-csp,graphicx,color}%from
\begin{document}
\begin{titlepage}
  \begin{figure}[h]
  \centerline{\small MAKERERE 
  \includegraphics[width=0.1\textwidth]{muk_log} UNIVERSITY}
\end{figure}
\centerline{COLLEGE OF COMPUTING AND INFORMATIC SCIENCES}
\paragraph{•}
\centerline{DEPARTMENT OF COMPUTER SCIENCE\\}
\paragraph{•}

\centerline{COURSEWORK: RESEARCH METHODOLOGY(BIT 2207)\\}
\paragraph{•}

\centerline{LECTURER: MR.ERNEST MWEBAZE}
\paragraph{•}

\centerline{TOPIC\\}A LITERATURE REVIEW ON THE  NEARBY  A GOOGLE PRODUCT:\\
\paragraph{•}
\centerline{COMPILED BY: \
 KIZITO ANDREW}
 \paragraph{•}
\centerline{STUDENT NUMBER :  216017100}
\paragraph{•}
\centerline{REGISTRATION NUMBER:16/U/6236/PS}
\paragraph{•}
\end{titlepage}
\pagenumbering{roman}
\newpage
\pagenumbering{arabic}
\section{INTRODUCTION}
Nearby Connections is a peer-to-peer networking API that allows apps to easily discover, connect to, and exchange data with nearby devices in real-time, regardless of network connectivity. Use Nearby Connections to create multiplayer experiences or share with friends offline.
Nearby Notifications is a new feature allowing developers to tie an app or website to a BLE beacon and create contextual notifications, even with no app installed.

\section{Background}

Nearby was founded by Brian Hamachek\cite{mcalexander2003using}. The company is located in Palo Alto, CA. The company is a member of the Microsoft Bizspark program.On November 11, 2013, Nearby was accepted into the Fall 2013 session of the Stanford StartX accelerator. In January 2014, the company name was changed from WNM Live to Nearby.

\section{Service overview.}

Nearby’s stated purpose is to help people make new friends. To accomplish this, the service uses the GPS unit in a phone or computer to determine your location and returns a list of users nearby based on relative proximity\cite{baumert2011system}. The service features private text messaging, photo messaging, virtual gifts, and profiles. There is also a public feed called “Live Stream” which is comparable to Facebook’s News Feed, et al.Unlike several other location-based social networks such as Skout, WhosHere, and Grindr, Nearby explicitly declares that the service is not intended to be used for dating purposes.\cite{doshi2014location}

\subsection{Platforms and users}

A Windows Mobile 6 application was released on 6/2010. It was followed by a Windows Phone 7 application which was released on 10/2010. A web and mobile web portal for the service was launched 4/2011. An iPhone application was released 1/2012.A Windows 8 application was released in 5/2012.] An Android application was released in the first half of 2013. The service has a combined membership\cite{zheng2013nearby} of just over 5 million users. The highest percentage of users are located in the United States, United Kingdom, and India (in that order). An Android app was released in January 2015.

\section{CONCLUSIONS.}
In conclusion, nearby is an android based app which enables resource (such as messages, notifications and gaming and file sharing) sharing amongst its users. \cite{harrell1997method}
\newpage
\bibliographystyle{IEEEtran}
\bibliography{REFRENCES}

\end{document} 
